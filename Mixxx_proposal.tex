\documentclass[a4paper,12pt]{scrartcl}
\usepackage[utf8x]{inputenc}
\usepackage[english]{babel}

\usepackage{url}
\usepackage[pdfauthor={Nazar ``tr0'' Gerasymchuk},%
pdftitle={Proposal for Mixxx's project ``Non-Blocking Database Access''},%
pagebackref=true,%
pdftex]{hyperref}

%opening
\title{Proposal for Mixxx's project ``Non-Blocking Database Access''}
\author{Nazar ``tr0'' Gerasymchuk, \\ \texttt{\normalsize Nazar.Gerasymchuk@gmail.com}}

\begin{document}

\maketitle
\begin{abstract}
I'm getting Master degree at Cybernetics Department of Taras Shevchenko National
University of Kyiv, Ukraine (\url{http://univ.kiev.ua/en}).

I'm Qt programmer for about 3 years. I have some experience with programming databases 
(MySQL, SQLite), threads etc.

I have burning desire to take a part in development of Mixxx. Also, I'm interested in music 
production. You can listen some of my own tracks here -- \url{http://soundcloud.com/tr0}.
Sometimes I write programming notes at my blog -- \url{http://neval8.wordpress.com}. 
Hope, soon there will be more themes on related to development of Mixxx.
\end{abstract}

\tableofcontents

\newpage
\section{Intro}

\textbf{``Non-Blocking Database Access''} is one of the clearest projects for me.

I have some skills and practice with programming databases using Qt. Also I have a theoretical base 
I've not used yet. I'm sure the Mixxx can be a right place to improve my programming skills and 
I see possibility to make my personal contribution into Open~Source in general and into 
Mixxx in particular with GSoC 2013 team.

\section{What do I have now?}
\begin{itemize}
 \item My working OS is Debian GNU/Linux now.
 \item I'm registered user at Launchpad and I'm able to check out Mixxx sources.
 \item I'm studying the \texttt{bzr} usage. I see that \texttt{bzr} is nice distributed revision 
    control system.
 \item I accomplished connecting \texttt{scons} to QtCreator, so I got my favorite IDE working 
    with \texttt{scons}. And I compiled it (it took about 10 minutes on 4 cores). Here is 
    draft article about how I done that: \\ \url{http://neval8.wordpress.com/2013/04/30/using-scons-with-qtcreator}.
\end{itemize}


\section{What'll I do next?}
\begin{itemize}
 \item According to the fact that Mixxx is written in C++, we can make some static analysis by 
    \texttt{cppcheck}%
    \footnote{Cppcheck (\url{http://cppcheck.sourceforge.net}) is a static analysis tool for C/C++
    code. Unlike C/C++ compilers and many other analysis tools, it does not detect syntax errors in 
    the code. Cppcheck primarily detects the types of bugs that the compilers normally do not detect. 
    The goal is to detect only real errors in the code (i.e. have zero false positives).},
 as I did and got a bunch of warnings. Here, I can learn Mixxx sources deeper.
 \item Fix some issues found by \texttt{cppcheck}. Maybe, it is not critical, but it helps to learn 
    sources and become more familiar with it.
 \item Fix some issues and bugs from bug-tracker.
 \item Work on my ideas and test them.
 \item Think out on my road-map at GSoC 2013.
\end{itemize}

Some time ago, Ryan have described precisely the problem with databases on mailing list:
\textit{ Today's approach of doing some operations on the GUI thread blocks Qt from processing events. 
This has implications on Mixxx's responsiveness because things like waveform rendering cannot do work 
while the Qt main thread is blocked attempting to read/write from the database. As Daniel mentioned on 
the waveform thread, sometimes normal, small library operations hog the main thread for up to 20 ms. 
This is enough to cause a dropped frame when rendering the waveform at a reasonable FPS. It also increases 
the overall latency of the \texttt{ControlObject} system when the control events are proxied through 
the Qt event queue. So database queries on the main thread can add to the latency of pressing a button 
/ slider / knob on the GUI. These are all motivating factors for moving database queries to a thread.}


\section{How do I see the problem?}

After exploring Mixxx sources I've found out that we have concrete DAO in GUI thread to access database. 
That concrete DAO applies query to database.

So, we need to keep all business logic the same, but bring all database queries beyond GUI thread. 
All what is needed is usage of bare lambdas (introduces in new C++11 standard) and Qt's \texttt{QtConcurrent::run}.

To implement user interaction while database is applying query, we should do next:
\begin{itemize}
 \item Avoid applying database queries from GUI thread. \textit{It should be \texttt{QtConcurrent} with 
    lambda which I propose.}
 \item We must agree on how UI will react while applying queries:
 \begin{itemize}
  \item What to do in case of ``quick'' query (for example, $< 200 ms$)?
  \item What to do in case of ``long'' query (for example, $\approx 3 s$)?
  \item \underline{Who}, \underline{when}, \underline{where} and \underline{how} will inform user 
    (for example, show \texttt{ProgressBar}, show \texttt{MessageBox} or so on)?
  \item Is button ``Cancel'' planned?
 \end{itemize}
\end{itemize}

\subsection{Transactions and multithreading}
Transactions are good mechanism, and all of queries which modify database can be wrapped onto transaction 
system. But, well-known fact is that SQLite able to hold multithreading, but doesn't like it.

Snazzer at StackOverflow%
\footnote{``How to use SQLite in a multi-threaded application?'', \url{http://stackoverflow.com/a/1680871}}
say on this:
Some steps when starting out with SQLlite for multi-threaded use:
\begin{enumerate}
 \item Make sure SQLite is compiled with the multi-threaded flag.
 \item You must call open on your SQLite file to create a connection on each thread, don't share connections 
    between threads.
 \item SQLite has a very conservative threading model, when you do a write operation, which includes opening 
    transactions that are about to do an INSERT/UPDATE/DELETE, other threads will be blocked until this 
    operation completes.
 \item If you don't use a transaction, then transactions are implicit, so if you start a 
    \texttt{INSERT/DELETE/UPDATE}, SQLite will try to acquire an exclusive lock, and complete the operation 
    before releasing it.
 \item If you do a \texttt{BEGIN EXCLUSIVE} statement, it will acquire an exclusive lock before doing operations 
    in that transaction. A \texttt{COMMIT} or \texttt{ROLLBACK} will release the lock.
 \item \ldots
\end{enumerate}

We must keep in mind that connection can only be used from within the thread that created it. Moving connections 
between threads or creating queries from a different thread are not supported%
\footnote{``Thread-Support in Qt Modules'' --
\url{http://qt-project.org/doc/qt-4.8/threads-modules.html\#threads-and-the-sql-module}}.

To avoid unexpected behaviour as result of multi-threaded queries, as I think, we must think out and create own 
query manager. And till that time, try to avoid such type of concurrent queries.

\paragraph{Library scanner and BaseSqlTableModel.} Unfortunately, I'm not involved into development of 
Mixxx as well. Hope, that if this project will be approved as GSoC project for this summer, I can learn all nuances
and propose something better.

There will be some part of work on \texttt{BaseSqlTableModel}. Creating own class by subclassing \texttt{QAbstractTableModel}
and extending it to hold SQL-table -- is good choice and it needs work on it and some kind of research. As I think that work 
will be prolific and successful. I included this task into road-map.

\subsection{How to use lambdas here?}

Lambdas%
\footnote{\textit{``Lambdas in C++''} at Wiki --
\url{http://en.wikipedia.org/wiki/Anonymous\_function\#C.2B.2B},
\textit{``What is a lambda expression in C++11?''} at StackOverflow --
\url{http://stackoverflow.com/questions/7627098/what-is-a-lambda-expression-in-c11}.}
can help not to mess the code in places where ``fix'' will be applied. We just move the code that is called 
after response arrives to the lambda and pass it to the \texttt{DAO} as callback parameter. In this case we 
do not hang the application, but can gently show ``wait'' message.


\subsection{QtConcurrent}
\texttt{QtConcurrent}%
\footnote{\textit{``QtConcurrent Namespace''} at QtProject -- \url{http://qt-project.org/doc/qt-4.8/qtconcurrent.html}.}
is simple mechanism to get programs multi-threaded with minimal overhead and also with minimal control on 
respective threads that is enough for us. Also, we can pass lambda to \texttt{QtCoucurrent::run} as parameter.

The \texttt{QtConcurrent} name space provides high-level APIs that make it possible to write multi-threaded 
programs without using low-level threading primitives such as mutexes, read-write locks, wait conditions, or 
semaphores. Programs written with \texttt{QtConcurrent} automatically adjust the number of threads used 
according to the number of processor cores available. This means that applications written today will continue 
to scale when deployed on multi-core systems in the future.

The \texttt{QtConcurrent::run()} function runs a function in a separate thread. The return value of the function 
is made available through the \texttt{QFuture} API.
Example:

\begin{verbatim}
    extern void aFunction();
    QFuture<void> future = QtConcurrent::run(aFunction);
\end{verbatim}
    
This will run \texttt{aFunction} in a separate thread obtained from the default \texttt{QThreadPool}. You can use the 
\texttt{QFuture} and \texttt{QFutureWatcher} classes to monitor the status of the function.

\texttt{QFuture} allows threads to be synchronized against one or more results which will be ready at a later point 
in time. The result can be of any type that has a default constructor and a copy constructor. 

The state of the computation represented by a \texttt{QFuture} can be queried using the \texttt{isCanceled()}, 
\texttt{isStarted()}, \texttt{isFinished()}, \texttt{isRunning()}, or \texttt{isPaused()} functions.

\texttt{QFuture<void>} is specialised to not contain any of the result fetching functions. Any \texttt{QFuture<T>} 
can be assigned or copied into a \texttt{QFuture<void>} as well. This is useful if only status or progress information 
is needed -- not the actual result data.

\section{How to solve problem?}
I propose to rewrite code of calling \texttt{DAO}-objects. I created minimal project to show what I recommend to do. 
Here, you can see it: \url{http://github.com/troyane/lambdaConcurrent}.

Main scheme is next:
\begin{itemize}
 \item In GUI:
 \begin{enumerate}
  \item Prepare query string (as it was before).
  \item Prepare lambda \textit{``how to prepare GUI for long-time operation''} \\(Inform user. Show some progress bar etc.).
  \item Prepare lambda \textit{``how to release GUI after long-time operation''} \\(Inform user. Hide some progress bar etc.).
  \item Call concrete \texttt{DAO}s \texttt{applyQuery} function with (2) and (3) parameters (lambdas 
    \textit{``how to prepare GUI for long-time operation''}, \textit{``how to release GUI after long-tim operation''}).
  \item Release GUI thread -- let it flow as it is.
 \end{enumerate}
 \item In Concrete \texttt{DAO}s \texttt{applyQuery}:
 \begin{enumerate}
  \item Apply own event filter to block user input.
  \item Wrap all code originally placed in concrete \texttt{DAO} into lambda.
  \item Send lambda work to other thread (using \texttt{QtConcurrent::run}).
  \item Control time of thread working (using \texttt{QFuture}).
  \begin{itemize}
   \item If thread is working longer than some constant limit time, we must apply received lambda as a parameter 
    \textit{``how to prepare GUI for long-time operation''}.
   \item If thread overtimed, than we need to apply received lambda as a parameter \textit{``how to release GUI after long-time operation''}.
  \end{itemize}
  \item Remove own event filter to unblock user input.
 \end{enumerate}
\end{itemize}

We'll wrap all code of \texttt{DAO}'s with the next construction (see \texttt{DAO::applyQuery} in file 
\url{http://github.com/troyane/lambdaConcurrent/blob/master/dao.cpp)}.

I tried to comment as clear as I can, but if you have questions -- you are welcome.


\section{Approximate road-map}
\begin{enumerate}
 \item Learn Mixxx sources (to understand how does it work) -- \textit{continuous process}.
 \begin{enumerate}
  \item Play with code.
  \item Create own brunch.
  \item Solve my initial problems with code on IRC channel.
 \end{enumerate}
 \item Fix bugs (to get involved into development process) -- \textit{continuous process}.
 \item Learn more about Qt, multi-threading, databases etc related to Mixxx development -- \textit{continuous process}.
 \item Implement my idea on concrete example.
 \begin{enumerate}
  \item Discuss idea on IRC, on mail list.
  \item Write code.
  \item Write documentation.
  \item Publish general scheme on how to implement code for database access.
 \end{enumerate}
 \item Rewrite all database access entries in Mixxx source to comply main scheme.
 \begin{itemize}
  \item Find all entries.
  \item Discuss entries.
  \item Make changes.
  \item Test applied changes.
 \end{itemize}
 \item Work on \texttt{BaseSqlTableModel}...
 \begin{itemize}
  \item Research how \texttt{BaseSqlTableModel} and etc work.
  \item Discuss possible changes to improve database access.
  \item Write code.
  \item Write documentation.
 \end{itemize}

 \item Discuss all changes.
 \item Perform general test.
 \item Work on general documentation.
 \item Code revise.
\end{enumerate}

\end{document}