\documentclass[a4paper,12pt]{scrartcl}
\usepackage[utf8x]{inputenc}
\usepackage[english]{babel}


\usepackage{url}
\usepackage[pdfauthor={Nazar ``tr0'' Gerasymchuk},%
pdftitle={Proposal for Mixxx's project ``Non-Blocking Database Access''},%
pagebackref=true,%
pdftex]{hyperref}

%opening
\title{Proposal for Mixxx's project ``Non-Blocking Database Access''}
\author{Nazar ``tr0'' Gerasymchuk, \\ \texttt{\normalsize Nazar.Gerasymchuk@gmail.com}}

\begin{document}

\maketitle
\nvDash
\begin{abstract}
Let me introduce myself. I'm getting Master degree at Cybernetics
Department of Taras Shevchenko National University of Kyiv, Ukraine (\url{http://univ.kiev.ua/en}).

I'm Qt programmer for about 3 years. I have an experience with programming databases (MySQL, SQLite), threads etc.

I still have burning desire to take part in development of Mixxx. Also, I'm interested in music production. You can listen some of my own tracks here -- \url{http://soundcloud.com/tr0}.
\end{abstract}

\tableofcontents

\section{Intro}

\textbf{``Non-Blocking Database Access''} is one of most clear projects for me.

I have some skills and practice with programming databases using Qt. So think that project is suitable for me. Sometimes I write programming notes at my blog -- \url{http://neval8.wordpress.com}. Hope, there will be more on related to development of Mixxx themes soon.

\section{What do I have now?}
\begin{itemize}
 \item My working OS is Debian GNU/Linux now.
 \item I'm registered user at Launchpad and am able to chek out Mixxx sources.
 \item I'm learning to use \texttt{bzr}. I see that \texttt{bzr} is nice distributed revision control system.
 \item I accomplished connecting \texttt{scons} to QtCreator, so I got my favorite IDE working with \texttt{scons}. 
 And I compiled it (it took about 10 minutes on 4 cores). Here is draft article about how I do that: \\ \url{http://neval8.wordpress.com/2013/04/30/using-scons-with-qtcreator}.
\end{itemize}


\section{What'll I do next?}
\begin{itemize}
 \item According to the fact of Mixxx is written in C++, we can make some static analysis by \texttt{cppcheck}%
 \footnote{Cppcheck (\url{http://cppcheck.sourceforge.net}) is a static analysis tool for C/C++ code. Unlike C/C++ compilers and many other analysis tools it does not detect syntax errors in the code. Cppcheck primarily detects the types of bugs that the compilers normally do not detect. The goal is to detect only real errors in the code (i.e. have zero false positives).},
 as I did and got a bunch of warnings. Here, I can learn Mixxx sources deeper.
 \item Fix some issues found by \texttt{cppcheck}. Maybe it is not critical, but it helps to learn sources and become more familiar with it.
 \item Fix some issues and bugs from bugtracker.
 \item Work on my ideas and test them.
 \item Think out on my roadmap at GSoC 2013.
\end{itemize}

Some time ago, Ryan described problem with databases well on mailing list:
\textit{ Today's approach of doing some operations on the GUI thread blocks Qt from processing events. This has implications on Mixxx's responsiveness because things like waveform rendering cannot do work while the Qt main thread is blocked attempting to read/write from the database. As Daniel mentioned on the waveform thread, sometimes normal, small library operations hog the main thread for up to 20 ms. This is enough to cause a dropped frame when rendering the waveform at a reasonable FPS. It also increases the overall latency of the \texttt{ControlObject} system when the control events are proxied through the Qt event queue. So database queries on the main thread can add to the latency of pressing a button / slider / knob on the GUI. These are all motivating factors for moving database queries to a thread.}


\section{How I see the problem?}

After exploring Mixxx sources I've found out that we have concrete DAO in GUI thread to access database. That concrete DAO applies query to database. Maybe, I missed something, but I haven't seen any errors processing (because a lot of things can happen while applying query) \texttt{// that is another story}.

So, we need to keep all business logic the same, but bring all database queries beyond GUI thread. All what is needed is usage of bare lambdas (introduces in new C++11 standard) and Qt's \texttt{QtConcurrent::run}.

\subsection{How to use lambdas here?}

Lambdas%
\footnote{\textit{``Lambdas in C++''} at Wiki -- 
\url{http://en.wikipedia.org/wiki/Anonymous\_function\#C.2B.2B}, 
\textit{``What is a lambda expression in C++11?''} at StackOverflow -- 
\url{http://stackoverflow.com/questions/7627098/what-is-a-lambda-expression-in-c11}.}
can help not to mess the code in places where ``fix'' will be applied. We just move the code that is called after response arrives to the lambda and pass it to the \texttt{DAO} as callback parameter. In this case we do not hang the application, but can gently show ``wait'' message. 


\subsection{QtConcurrent}
\texttt{QtConcurrent}%
\footnote{\textit{``QtConcurrent Namespace''} at QtProject -- \url{http://qt-project.org/doc/qt-4.8/qtconcurrent.html}.}
is simple mechanism to get programs be multi-threaded with minimal overhead and also with minimal control on respective threads that is enough for us. Also, we can pass lambda to \texttt{QtCoucurrent::run} as parameter.

To implement user interaction while database is applying query, we should do next:
\begin{itemize}
 \item Avoid applying database queries from GUI thread. \textit{It should be \texttt{QtConcurrent} with lambda which I propose.}
 \item We must agree on how UI will react while applying queries:
 \begin{itemize}
  \item What to do in case of ``quick'' query (for example, $< 200 ms$)?
  \item What to do in case of ``long'' query (for example, $\approx 3 s$)?
  \item \underline{Who}, \underline{when}, \underline{where} and \underline{how} will inform user (for example, show \texttt{ProgressBar}, show \texttt{MessageBox} or so on)?
  \item Is button ``Cancel'' planned?
 \end{itemize}
\end{itemize}

\section{How to solve problem?}
I propose to rewrite code of calling \texttt{DAO}-objects. I created minimal project to show what I recommend to do. Here, you can see it: \url{http://github.com/troyane/lambdaConcurrent}.

Main scheme is next:
\begin{itemize}
 \item In GUI:
 \begin{enumerate}
  \item Prepare query string (as it was before).
  \item Prepare lambda \textit{``how to freeze GUI''} \\(Inform user. Show some progress bar etc.).
  \item Prepare lambda \textit{``how to unfreeze GUI''} \\(Inform user. Hide some progress bar etc.).
  \item Call concrete \texttt{DAO}s \texttt{applyQuery} function with (2) and (3) parameters (lambdas \textit{``how to freeze GUI''}, \textit{``how to unfreeze GUI''}).
  \item Release GUI thread -- let it flow as it is.
 \end{enumerate}
 \item In Concrete \texttt{DAO}s \texttt{applyQuery}:
 \begin{enumerate}
  \item Apply own event filter to block user input.
  \item Wrap all code originally placed in concrete \texttt{DAO} into lambda.
  \item Send lambda work to other thread (using \texttt{QtConcurrent::run}).
  \item Control time of thread working (using \texttt{QFuture}).
  \begin{itemize}
   \item If thread is working longer than some constant limit time, we must apply lambda got as parameter \textit{``how to freeze GUI''}.
   \item If thread overtimed, than we need to apply lambda got as parameter \textit{``how to unfreeze GUI''}.
  \end{itemize}
  \item Remove own event filter to unblock user input.
 \end{enumerate}
\end{itemize}

We'll wrap all code of \texttt{DAO}'s with next construction (see \texttt{DAO::applyQuery} in file \url{http://github.com/troyane/lambdaConcurrent/blob/master/dao.cpp)}.

I tried to comment as clear as I can, but if you have questions -- you are welcome.


\section{Approximate roadmap}
\begin{enumerate}
 \item Learn Mixxx sources (to understand how does it work) -- \textit{continuous process}.
 \begin{enumerate}
  \item Play with code.
  \item Create own brunch.
  \item Solve my initial problems with code on IRC channel.
 \end{enumerate}
 \item Fix bugs (to get involved into development process) -- \textit{continuous process}.
 \item Learn more about Qt, multi-threading, databases etc related to Mixxx development -- \textit{continuous process}.
 \item Implement my idea on concrete example.
 \begin{enumerate}
  \item Discuss idea on IRC, on mail list.
  \item Write code.
  \item Write documentation.
  \item Publish general scheme on how to implement code for database access.
 \end{enumerate}
 \item Rewrite all database access entries in Mixxx source to comply main scheme.
 \begin{itemize}
  \item Find all entries.
  \item Discuss entries.
  \item Make changes.
  \item Test applied changes.
 \end{itemize}
 \item Discuss all changes.
 \item Perform general test.
 \item Work on general documentation.
 \item Code revise.
\end{enumerate}

\end{document}
